% !TeX spellcheck = en_GB
\documentclass{article}
\usepackage{graphicx}
\usepackage{titlesec}
\usepackage{array}
\usepackage[table]{xcolor}
\usepackage{color, colortbl}
\usepackage{stackengine}
\usepackage{fancyhdr}
\usepackage{longtable}
\usepackage{changepage}
\usepackage{tikz, graphicx}
\usepackage{hyperref}
\usepackage{listings}
\usepackage{float}

\newcommand\xrowht[2][0]
{\addstackgap[.5\dimexpr#2\relax]{\vphantom{#1}}}

\setlength{\arrayrulewidth}{0.4mm}
\setlength{\tabcolsep}{20pt}
\renewcommand{\arraystretch}{1.6}
\setcounter{tocdepth}{4}
\setcounter{secnumdepth}{4}

\pagestyle{fancy}
\renewcommand{\headrulewidth}{2pt}
\renewcommand{\footrulewidth}{1pt}

\begin{document}
\begin{titlepage}
	\begin{center}
		\begin{figure}
			\centering
			\includegraphics[scale=0.3]{LogoPolimi.PNG} \\
			[0.3cm]
		\end{figure}
		\Huge{\bfseries - DD -} \\
		[0.5cm]
		\huge{Design Document} \\
		[1mm]
		\rule{300pt}{3pt} \\
		[1.0cm]
		\textsc{\Large Computer Science and Engineering} \\ 
		\textsc{\Huge Software Engineering II} \\
		\textsc{\Large A.A. 2020/2021} \\
		[1cm]
		\textsc{\LARGE Daniele Mammone - 10625264} \\
		\textsc{\LARGE Gianmarco Naro - 10610374} \\
		\textsc{\LARGE Massimo Parisi - 10583470} \\
	\end{center}
\end{titlepage}

\newpage

\renewcommand\contentsname{Contents}
\tableofcontents

\newpage

\section{Introduction}

	\subsection{Purpose}
	The main target of this document is to describe the {\bfseries Customer Line-Up} (\emph{CLup}) design from a more detailed point of view. This document follows faithfully what was defined in the Requirement Analysis and Specification Document and must be read carefully before starting the software implementation in order to understand the software design in detail. Moreover, we tried to maintain independence between \emph{DD} document and \emph{RASD} document in order to allow greater flexibility in case you want to reuse the design model.
	
	\subsection{Scope}
	\emph{CLup} is a booking application and nowadays this type of applications are more and more widespread in people's everyday life and are becoming more and more indispensable. For this reason, intelligent design choices have been made that extrapolate the positive aspects of existing booking apps, in order to make CLup really “achievable” in real life.
	
	The main purpose of CLup is to facilitate customers to access at a store in {\bfseries security}, both allowing them to reserve a spot on the queue for entering the store through the app and to book a visit at the store at a determined time of a certain day, selected by the customer. Thanks to this, store managers can manage the {\bfseries affluence} in their store more easily, and moreover can reduce the crowd in front of the store, that is one of the main purposes of the application. Another important feature deals with obtaining statistics from customers and generic information from stores in order to help customers during their reservation. For this, the system has to store a very large amount of data. This data are mined and used later, which is why it is very important to identify the user's role so that we can provide him with more accurate and suitable information. Indeed a customer wants to provide information about the stores while, for example, a store manager is more interested in information about customer permanence in the store.
	
	\subsection{Definitions, Acronyms, Abbreviations}
	\bigskip
		\subsubsection{Definitions}
			\begin{center}
				\rowcolors{2}{}{gray!20}
				\rowcolors{1}{gray!20}{white}
				\renewcommand{\arraystretch}{2.5}
				
				\begin{adjustwidth}{-2.3cm}{}
					\begin{tabular}[h!]{|m{8em}|m{31em}|}
						\hline
						\xrowht{5pt}
						QR Code & Bi dimensional bar code that allows the user to check-in/check-out at the store entries/exits \\
						\xrowht{5pt}
						Reservation & Indicates both booked visits and spots on the queue to enter the store as soon as possible \\
						\xrowht{5pt}
						Customer & The clients of the store, that uses the system to get a reservation to access the store \\
						\xrowht{5pt}
						Store manager & The app user that access to stores’ bookings, occupancy and settings, in order to manage the flow of customers \\
						\xrowht{5pt}
						QR Code Reader & Device used to scan customers’ \emph{QR Code} \\
						\xrowht{5pt}
						Totem & Electronic device that allows customers to physically get a spot on the queue to enter the store a soon as possible; it allows to specify the same parameters that can be inserted through the app \\
						\xrowht{5pt}
						QR Code Printer & Device used by totems to print \emph{QR Code} \\
						\xrowht{5pt}
						Department & Part of the store that contains the same category of products \\
						\xrowht{5pt}
						Query & Synonym for request \\
						\hline
					\end{tabular}
				\end{adjustwidth}
			\end{center}
		\subsubsection{Acronyms}
			\begin{center}
				\rowcolors{2}{}{gray!20}
				\rowcolors{1}{gray!20}{white}
				\renewcommand{\arraystretch}{2.5}
				
				\begin{adjustwidth}{-2.3cm}{}
					\begin{tabular}[h!]{|m{4em}|m{35em}|}	
						\hline
						\xrowht{5pt}
						\centering RASD & Requirement Analysis and Specification Document \\
						\xrowht{5pt}
						\centering DD & Design Document \\
						\xrowht{5pt}
						\centering ETA & Estimated Time of Arrival \\
						\xrowht{5pt}
						\centering GPS & Global Positioning System \\
						\xrowht{5pt}
						\centering API & Application Programming Interface \\
						\xrowht{5pt}
						\centering UML & Unified Modeling Language \\
						\xrowht{5pt}
						\centering DBMS & DataBase Management Service \\
						\xrowht{5pt}
						\centering OS & Operative System \\
						\xrowht{5pt}
						\centering HTTPS & HyperText Transfer Protocol over Secure Socket Layer \\
						\xrowht{5pt}
						\centering TCP & Trasmissione Control Protocol \\
						\xrowht{5pt}
						\centering IP & Internet Protocol \\
						\hline	
					\end{tabular}
				\end{adjustwidth}
			\end{center}
		\bigskip
		\subsubsection{Abbreviations}
			\begin{center}
				\rowcolors{2}{}{gray!20}
				\rowcolors{1}{gray!20}{white}
				\renewcommand{\arraystretch}{2.5}
				
				\begin{adjustwidth}{-2.3cm}{}
					\begin{tabular}[h!]{|m{4em}|m{35em}|}
						\hline
						\xrowht{5pt}
						\centering IIT & Implementation, Integration and Testing \\
						\xrowht{5pt}
						\centering Rn & Requirement number n \\
						\xrowht{5pt}
						\centering ASAP & As soon as possible \\
						\hline
					\end{tabular}
				\end{adjustwidth}
			\end{center}
	\subsection{Revision History}
		\begin{center}
			
			\renewcommand{\arraystretch}{1.5}
			\begin{adjustwidth}{-2.3cm}{}
				\begin{tabular}[h!]{|m{4em}|m{5em}|m{26em}|}	
					\hline
					\rowcolor{gray!20}
					\xrowht{5pt}
					\centering Version & \centering Date & Changelog \\
					\hline
					\xrowht{5pt}
					\centering 1.0 & 29/11/2020 & First version \\
					\hline	
				\end{tabular}
			\end{adjustwidth}
			
		\end{center}
	\subsection{Software and Tools}
		\begin{itemize}
			\item {\LaTeX} as software system for docuement preparation
			\item UMLet for the UML diagrams and other diagrams
			\item Photoshop for the mockups
			\item Git \& Github as work space. The repository is \href{https://github.com/danmaam/MammoneNaroParisi}{here}.
		\end{itemize}
	\subsection{Reference Documents}
		\begin{itemize}
		\item Specification Document
		\item Slides of the lectures
		\end{itemize} 
	\subsection{Document Structure}
		The structure of the document is thought with the intention of allowing simple navigation through it. Also, various abbreviations, highlighted in Abbreviations section, have been used to make the content smoother.
		Hence, the structure of the document is the following one:
		\begin{itemize}
			\item {\bfseries Introduction}: this section gives a general view of the problem and describes the scope and purpose of the \emph{DD}, including a set of definitions, acronyms and abbreviations used.
			
			\item {\bfseries Architectural Design}: this section starts with a high level description of the architecture of the system and continues going into details, specifying the components and interfaces used.
			
			\item {\bfseries User Interface Design}: this section presents the mockups of the application, describing how the clients can navigate the application, highlighting the actions they can do.
			
			\item {\bfseries Requirements Traceability}: this section describes the connection between the RASD and the DD, identifying the relation between goals and requirements described previously and the components that allow to realize them.
			
			\item {\bfseries Implementation, Integration and Test Plan}: this section establishes a plan for the development of components, identifying the conditions needed to be met before starting the development process and then maximizing the efficiency of the developer teams with a precise schedule.
			
			\item {\bfseries Effort Spent}: the main focus of this section is to track the time spent to complete this project. In particular, is highlighted the subdivision of the working hours of the various sections.
			
			\item {\bfseries References}: this section is dedicated to all references used in this project.
		\end{itemize}

\section{Architectural Design}
	The aim of this section is to give a look to the architectural aspect of the analysed system. In doing this, a top-down approach will be followed, starting from a very general view of the system, then going into more detail through Component Diagrams, more and more detailed. This allows us to precisely describe each component of CLup, so that who will develop the system, well know the behaviour of each component.
	
	
	\subsection{Overview}
	A first high level overview can be done through the system's \textbf{Composition Diagram}. In fact, here there is a first subdivision of the system's component, each one with a different task.
	\begin{itemize}
		\item \textbf{Communication Interfaces}: to work properly, the system needs to interface overt the network with both remote applications, and external services. Because of this, there are some components devoted to interact with remote devices:
		\begin{itemize}
			\item{\bfseries UserHandler}: to interact with users, both Managers and Customers, the latter both at Totems, and on app;
			\item{\bfseries EntranceManagement}: so that the Store can process a scanned QR Code at entry/exit;
			\item{\bfseries NumberCallingInterface}: to notify the call of a certain number;
		\end{itemize}
		and others deputy to access external services, such as 
		\begin{itemize}
			\item{\bfseries Map Component}: to access an external Map Provider
			\item{\bfseries Notification Component}: to send notifications to Customers through an external Notification service 
			\item{\bfseries Mail Component}: to externally sending Mails, avoiding building an internal Mail manager
		\end{itemize}
	\begin{figure}[!h]
		\centering
		\includegraphics[scale=0.5]{../UML Diagrams/CompositionDiagram/CompositionDiagram.png} \\
		\caption{\emph{Number calling system}}
	\end{figure}
	
		\item{\bfseries Business Logic}: here there is the whole \emph{Business Logic} of the system. Since each store may maintain a different logic from another, there is a \emph{Component} for each store managed by the system. Each store, manages the following aspects:
		\begin{itemize}
			\item{\bfseries Reservation Manager}: manages all the things concerning \emph{reservations} and their making. It's useful also to retrieve some important data, such as ETAs and available time intervals to enter the store;
			\item{\bfseries Statistic Manager}: it's the part used for the calculation of statistics about customers' shopping time, and about the average time spent in a single department. Since each store is different from the other, the statistics are maintained unique per store. This part is also used by \textbf{Reservation Manager} to calculates ETAs and time intervals;
			\item{\bfseries Number Calling System}: is the component dedicated to admit reserved customers inside the store;
			\item{\bfseries Building Flow Manager}: it takes care of processing scanned QR Codes at the entry and the exit of the store;
			\item{\bfseries Store Settings Manager}: manages all the things concerning stores' settings and parameters, such as working hours and capacity (also for each department).
			
		\end{itemize} This subdivision allows to separate better the logic off each store. Moreover, there are component common to each store, so the ones deputy to allow users to request services to the system.
		\begin{itemize}
			\item {\bfseries Access Manager}: manages users' \emph{registraion} and \emph{log-in};
			\item {\bfseries Suggestion Manager}: manages the retrieve of \emph{suggestions} when required, or necessary;
			\item{\bfseries Customer Alert Service}: takes care of sending departing notifications to customers.
		\end{itemize}
	At the end, there is another block: the \textbf{Data Interface}: it represents the \emph{Data Tier} of the system and separates the Business Logic from the data. It interacts with the DBMS to store and load informations.
	\end{itemize}

	
	\subsection{Component View}
	\subsection{Deployment View}
	\subsection{Runtime View}
	\subsection{Component Interfaces}
	\subsection{Selected Architectural Styles and Patterns}
	\subsection{Other design decisions}

\section{User Interface Design}

\section{Requirements Tracebility}

\section{Implementation, Integration and Test Plan}

\section{Effort Spent}

\section{References}
\end{document}